\documentclass[10pt,a4paper]{book}

\usepackage{lipsum}

\usepackage{venturis2}
\usepackage[T1]{fontenc}

\usepackage{fancyhdr}
\pagestyle{fancy}

\lhead{\leftmark}
\rhead{\rightmark}
\cfoot{\thepage}

\newenvironment{dedication}
{
   \clearpage
   \thispagestyle{empty}
   \vspace*{\stretch{1}}
   \itshape
   \raggedright
}%
{
   \par
   \vspace*{\stretch{3}}
   \clearpage
}

\begin{document}

\frontmatter

\title{Fly Software Process}
\author{Nguyen Anh Quan\\
  \texttt{naquan2112@gmail.com}}
\date{\today}
\maketitle

\begin{dedication}
  I *believed I can fly.
\end{dedication}

\tableofcontents

\mainmatter

\chapter{Introduction}

First of all, I am no MA or doctor, I am a developer who however many times
changing job can not found a satisfying software process.

In this book, I will share with you the concepts and details of what I *think
is software process that give people who participate both freedom and joy, like
when flying, thus the name. Most of these ideas are already rejected as MA
thesis If you agree with these ideas, please help me build this. If you
disagree with these altogether, please ignore this book, it 's definitely not
for you.

This book is free(as in freedom) and by using it, you owns me nothing, you have
all the rights the General Public License(GPL) license gives you. Feel free to feed back in Github 's
bug reporting section. And you 're also welcomed to contribute to the writing
of this book also, I will be glad seeing your pull requests on Github.

\chapter{Concepts}
\section[Learning Process]{Fact, Hypothesis, and Conclusion - the Learning Process}

I did think of this when I 'm in Artificial Intelligent class at Ho Chi Minh
University of Natural Science. Back then, my professor is explaining about ?ref
a knowledge model. It start with rules that 's represented as a logic rule:

% make mathematics formula
F1 && F2 -> S3
\lipsum[1-20]

\section[Governing Forces]{Knowledge, Money, and Power - the Governing Forces}

\lipsum[1-20]

\section[Working Together]{Cooperation, Collaboration, and Co-construction - Working Together}

\lipsum[1-20]

\backmatter

\end{document}
